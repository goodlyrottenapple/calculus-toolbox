\documentclass[crop,tikz,convert={outext=.svg,command=\unexpanded{pdf2svg \infile\space\outfile}},multi=false]{standalone}[2012/04/13]
%\usetikzlibrary{...}% tikz package already loaded by 'tikz' option
\usepackage[latin1]{inputenc}
\usepackage{tikz}
\usetikzlibrary{shapes,arrows, positioning}
\begin{document}
\pagestyle{empty}


% Define block styles
\tikzstyle{block} = [rectangle, draw,
    text width=10em, text centered, rounded corners, minimum height=3em]
\tikzstyle{line} = [draw, -latex']
    
\begin{tikzpicture}[node distance =2cm and 6mm, auto]
    % Place nodes
    \node [block] (json) {Load calculus description file};

    \node [block, below of=json] (parse1) {Generate parser class for core calculus};
    \node [block, left= of parse1] (isa1) {Generate core calculus Isabelle theory};
    \node [block, right= of parse1] (print1) {Generate print class for core calculus};

    \node [block, below of=isa1] (scala1) {Export Scala version of core calculus};

    \node [block, right= of scala1] (comp1) {Compile Scala classes};

    \node [block, below of=scala1] (rule) {Parse calculus rules into Isabelle};

    \node [block, below of=rule] (isa2) {Generate calculus rules Isabelle theory};

    \node [block, right= of isa2] (parse2) {Rebuild parser class for full calculus};
    \node [block, right= of parse2] (print2) {Rebuild print class for full calculus};

    \node [block, below of=isa2] (scala2) {Export Scala version of full calculus};

    \node [block, right= of scala2] (comp2) {Compile Scala classes};

    \node [block, below of=comp2] (gui) {Generate Scala UI classes};

    % Draw edges
    \path [line] (json) -- (isa1);
    \path [line] (json) -- (parse1);
    \path [line] (json) -- (print1);

    \path [line] (isa1) -- (scala1);

    \path [line] (scala1) -- (comp1);
    \path [line] (parse1) -- (comp1);
    \path [line] (print1) -- (comp1);

    \path [line] (comp1) -- (rule);

    \path [line] (rule) -- (isa2);

    \path [line] (isa2) -- (scala2);

    \path [line] (scala2) -- (comp2);
    \path [line] (parse2) -- (comp2);
    \path [line] (print2) -- (comp2);

    \path [line] (comp2) -- (gui);

    % \path [line] (identify) -- (evaluate);
    % \path [line] (evaluate) -- (decide);
    % \path [line] (decide) -| node [near start] {yes} (update);
    % \path [line] (update) |- (identify);
    % \path [line] (decide) -- node {no}(stop);

\end{tikzpicture}
\end{document}